\documentclass[twocolumn]{article}
\usepackage{graphicx}
\usepackage{nameref}
\usepackage{amsmath}
\usepackage{amsfonts}
\usepackage{amssymb}
\usepackage{hyperref}
\usepackage{cleveref}
\usepackage{setspace}
\graphicspath{{./figures}}

\onehalfspacing

\begin{document}

\title{Heart Condition Diagnosing via Machine Learning}
\author{
  Champagne, Matthew \\
  \texttt{matthew.champagne@stonybrook.edu}
  \and
  Khadka, Abiral \\
  \texttt{abiral.khadka@stonybrook.edu}
  \and
  Flores, Yasmin \\
  \texttt{yasmin.flores@stonybrook.edu}
}

\maketitle
\section{Abstract} 


\section{Key Words} 
KEYWORDS: Machine Learning, Heart Disease, Electronic Stethoscope, Proactive Monitoring, Cardiovascular Disease Management, Telehealth 

\section{Introduction}
yasmin to work on 

\section{Background and Motivation} 
In this section, we first provide background knowledge on the importance of monitoring heart related problems 
and one of the most common standard methods used in heart disease evaluation via the usage 
of a stethoscope. We then introduce our new approach on a stethoscope design and its limitations 
of directly capturing a range of datasets.

\section{Importance of Monitoring Heart Related Problems} 
Monitoring heart disease problems is essential for early detection, effective management and prevention of serious complications. 
According to the American Heart Association, regular monitoring assists in identifying risk factors such as: 
‘diet quality, physical activity, smoking, body mass index, blood pressure, total cholesterol, blood glucose and sleep quality’ [X]. 
In other words, regular monitoring not only allows the doctors to treat the patients’ health 
but also make the necessary adjustments to optimize the treatment plan by considering several risk factors 
and improve long-term results. Ultimately, monitoring of heart health can be viewed as a proactive method 
in lowering heart problems while also serving as an essential method to continuously better prognose patients 
who already have existing heart related problems. 

\section{Traditional Stethoscopes} 
One of the main instruments in diagnosing and monitoring heart related problems is a stethoscope as 
seen in figure 2. This medical tool is composed of three parts: a chestpiece, tubing and 
a set of earpieces. It then functions by amplifying internal sounds from the body through two 
important elements: vibrations and sound waves [2]. It works when the chestpiece/diaphragm is placed on the 
patient’s chest, where the heartbeat creates soundwaves that makes the chestpiece to vibrate. Where afterwards these 
vibrations make its way through the tubing and into the earpieces. At this point the doctor 
can begin to interpret the heartbeat and sounds. As shown in figure 3, there are four 
common breathing sounds when using a stethoscope which all can be interpreted when diagnosing heart related problems. 

\section{DATA COLLECTED}  
The sound data was collected using a 3M Littmann Electronic Stethoscope model 3200, positioned on specific chest zones divided into upper, middle, and lower sections on both the left and right sides, including anterior and posterior locations. The stethoscope transmitted sound data to a computer via Bluetooth, and the 3M Littmann  StethAssist Visualization software was used to extract recordings in .wav format. The recordings were filtered through three modes (Bell, Diaphragm, and Extended) to emphasize different frequency ranges and highlight specific sound profiles.
The dataset contained 112 participants aged 12 to 90 years including 43 females and 69 males. Among these, 35 were healthy, while 77 had respiratory conditions such as asthma (32), pneumonia (5), COPD (9), bronchitis (3), heart failure (21), lung fibrosis (5), and pleural effusion (2). Each participant contributed a single recording lasting 5–30 seconds from specific chest zones. The data files included annotations detailing health conditions, sound types, chest zones, and demographic information, making this dataset a valuable resource for developing algorithms for detecting and diagnosing pulmonary diseases.


\cite{heart}
\cite{telehealth}

\bibliographystyle{plain}
\bibliography{ESE_534_Report}

\appendix

\end{document}
\endinput

